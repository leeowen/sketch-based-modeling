%% This is file `cag-template.tex',
%% 
%% Copyright 2017 Elsevier Ltd
%% 
%% This file is part of the 'Elsarticle Bundle'.
%% ---------------------------------------------
%% 
%% It may be distributed under the conditions of the LaTeX Project Public
%% License, either version 1.2 of this license or (at your option) any
%% later version.  The latest version of this license is in
%%    http://www.latex-project.org/lppl.txt
%% and version 1.2 or later is part of all distributions of LaTeX
%% version 1999/12/01 or later.
%% 
%% The list of all files belonging to the 'Elsarticle Bundle' is
%% given in the file `manifest.txt'.
%% 
%% Template article for Elsevier's document class `elsarticle'
%% with harvard style bibliographic references
%%
%% $Id: cag-template.tex 88 2017-06-09 07:40:26Z rishi $
%%
%% Use the options `twocolumn,final' to obtain the final layout
%% Use `longtitle' option to break abstract to multiple pages if overfull.
%% For Review pdf (With double line spacing)
%\documentclass[times,twocolumn,review]{elsarticle}
%% For abstracts longer than one page.
%\documentclass[times,twocolumn,review,longtitle]{elsarticle}
%% For Review pdf without preprint line
%\documentclass[times,twocolumn,review,nopreprintline]{elsarticle}
%% Final pdf
%\documentclass[times,twocolumn,final]{elsarticle}
%%
\documentclass[times,twocolumn,final]{elsarticle}
%%


%% Stylefile to load CAG template
\usepackage{cag}
\usepackage{framed,multirow}

%% The amssymb package provides various useful mathematical symbols
\usepackage{amssymb}
\usepackage{latexsym}

% Following three lines are needed for this document.
% If you are not loading colors or url, then these are
% not required.
\usepackage{url}
\usepackage{xcolor}
\definecolor{newcolor}{rgb}{.8,.349,.1}

\usepackage{hyperref}

\usepackage[switch,pagewise]{lineno} %Required by command \linenumbers below

\journal{Computers \& Graphics}

\begin{document}

\verso{Preprint Submitted for review}

\begin{frontmatter}

\title{Computers and Graphics submission formatting guidelines\tnoteref{tnote1}}%
\tnotetext[tnote1]{Only capitalize first
word and proper nouns in the title.}

\author[1]{Michael \snm{Wimmer}\corref{cor1}}
\cortext[cor1]{Corresponding author: 
  Tel.: +0-000-000-0000;  
  fax: +0-000-000-0000;}
\author[1]{Joaquim \snm{Jorge}\fnref{fn1}}
\fntext[fn1]{Editor-in-Chief, Computers and Graphics Journal.}  
\author[2]{Ross \snm{Laman}}
%% Third author's email
\ead{R.Laman@elsevier.com}
\author[2]{Gail \snm{Rodney}}

\address[1]{TU Wien, Favoritenstrasse 9-11/186, 1040 Wien, Austria}
\address[2]{INESC-ID, Rua Alves Redol, 9 1000-021 Lisboa, Portugal}

%\received{1 February 2017}
\received{\today}
%%%% Do not use the below for submitted manuscripts
%\finalform{28 March 2017}
%\accepted{2 April 2017}
%\availableonline{15 May 2017}
%\communicated{S. Sarkar}


\begin{abstract}
%%%
Example of an abstract: A biometric sample collected in an
uncontrolled outdoor environment varies significantly from its
indoor version. Sample variations due to outdoor environmental
conditions degrade the performance of biometric systems that
otherwise perform well with indoor samples. In this study, we
quantitatively evaluate such performance degradation in the case
of a face and a voice biometric system. We also investigate how
elementary combination schemes involving min-max or z
normalization followed by the sum or max fusion rule can
improve performance of the multi-biometric system. We use
commercial biometric systems to collect face and voice samples
from the same subjects in an environment that closely mimics the
operational scenario. This realistic evaluation on a dataset of
116 subjects shows that the system performance degrades in
outdoor scenarios but by multimodal score fusion the
performance is enhanced by 20\%. We also find that max rule
fusion performs better than sum rule fusion on this dataset. More
interestingly, we see that by using multiple samples of the same
biometric modality, the performance of a unimodal system can
approach that of a multimodal system.
%%%%
\end{abstract}

\begin{keyword}
%% MSC codes here, in the form: \MSC code \sep code
%% or \MSC[2008] code \sep code (2000 is the default)
%\MSC 41A05\sep 41A10\sep 65D05\sep 65D17
%% Keywords
\KWD Computers and Graphics\sep Formatting\sep Guidelines
\end{keyword}

\end{frontmatter}

\linenumbers

%% main text
\section{Note}
\label{sec1}
Please use \verb+elsarticle.cls+ for typesetting your paper.
Additionally load the package \verb+cag.sty+ in the preamble using
the following command: 
\begin{verbatim} 
  \usepackage{cag}
\end{verbatim}

Following commands are defined for this journal which are not in
\verb+elsarticle.cls+. 
\begin{verbatim}
  \received{}
  \finalform{}
  \accepted{}
  \availableonline{}
  \communicated{}
\end{verbatim}

Any instructions relevant to \verb+elsarticle.cls+ are applicable
here as well. See the online instruction available on:
\makeatletter
\if@twocolumn
\begin{verbatim}
 http://support.stmdocs.in/wiki/
 index.php?title=Elsarticle.cls
\end{verbatim}
\else
\begin{verbatim}
 http://support.stmdocs.in/wiki/index.php?title=Elsarticle.cls
\end{verbatim}
\fi

\subsection{Entering text}
\textcolor{newcolor}
{\bf Please note that there are no set page limits for this journal,
but we do ask that the length of the paper matches the paper's
contribution. Survey papers are welcome within or outside Special
Sections. High-resolution versions of images are welcome as
supplementary material, when they are crucial to better understanding
the image content and the paper's contribution.}

\section{The first page}
Avoid using abbreviations in the title. Next, list all authors with
their first names or initials and surnames (in that order). Indicate
the author for correspondence (see elsarticle documentation).

Present addresses can be inserted as footnotes. After having listed all
authors' names, you should list their respective affiliations. Link
authors and affiliations using superscript lower case letters.

\subsection{The Abstract}
An Abstract is required for every paper; it should succinctly summarize
the reason for the work, the main findings, and the conclusions of the
study. The abstract should be no longer than 200 words. Do not include
artwork, tables, elaborate equations or references to other parts of
the paper or to the reference listing at the end. ``Comment'' papers
are exceptions, where the commented paper should be referenced in full
in the Abstract.

The reason is that the Abstract should be understandable in itself to
be suitable for storage in textual information retrieval systems.

\textit{Example of an abstract: A biometric sample collected in an
uncontrolled outdoor environment varies significantly from its
indoor version. Sample variations due to outdoor environmental
conditions degrade the performance of biometric systems that
otherwise perform well with indoor samples. In this study, we
quantitatively evaluate such performance degradation in the case
of a face and a voice biometric system. We also investigate how
elementary combination schemes involving min-max or z
normalization followed by the sum or max fusion rule can
improve performance of the multi-biometric system. We use
commercial biometric systems to collect face and voice samples
from the same subjects in an environment that closely mimics the
operational scenario. This realistic evaluation on a dataset of
116 subjects shows that the system performance degrades in
outdoor scenarios but by multimodal score fusion the
performance is enhanced by 20\%. We also find that max rule
fusion performs better than sum rule fusion on this dataset. More
interestingly, we see that by using multiple samples of the same
biometric modality, the performance of a unimodal system can
approach that of a multimodal system.}

\section{The main text}

Please divide your article into (numbered) sections (You can find the
information about the sections at
\url{http://www.elsevier.com/wps/find/journaldescription.cws_home/505619/authorinstructions}).
Ensure that all tables, figures and schemes are cited in the text in
numerical order. Trade names should have an initial capital letter, and
trademark protection should be acknowledged in the standard fashion,
using the superscripted characters for trademarks and registered
trademarks respectively. All measurements and data should be given in
SI units where possible, or other internationally accepted units.
Abbreviations should be used consistently throughout the text, and all
nonstandard abbreviations should be defined on first usage
\citep{Vehlowetal2013}.

\begin{table*}[!t]
\caption{\label{tab1}Summary of different works pertaining to face and
speech fusion}
\centering
\begin{tabular}{|p{2.25cm}|p{2cm}|l|p{4cm}|p{3cm}|p{2cm}|}
\hline
Study & Algorithm used & DB Size & Covariates of interest & 
Top individual performance & Fusion\newline Performance\\
\hline
UK-BWG
(Mansfield et al.,
2001) &
Face, voice:\newline
Commercial & 200 & Time: 1--2 month\newline
separation (indoor) & 
TAR$^*$ at 1\% FAR$^{\#}$\newline
Face: 96.5\%\newline
Voice: 96\%
& --\\
\hline
Brunelli
(Brunelli and
Falavigna, 1995) & 
Face:\newline
Hierarchical\newline
correlation\newline
Voice:\newline
MFCC & 
87 & 
Time: 3 sessions, time\newline
unknown (indoor) & 
Face:\newline
TAR = 92\% at\newline
4.5\% FAR\newline
Voice:\newline
TAR = 63\% at\newline
15\% FAR
&
TAR =98.5\%\newline
at 0.5\% FAR\\
\hline
Jain
(Jain et al., 1999)
&
Face:\newline
Eigenface\newline
Voice:\newline
Cepstrum\newline
Coeff. Based
&
50
&
Time: Two weeks (indoor)
&
TAR at 1\% FAR\newline
Face: 43\%\newline
Voice: 96.5\%\newline
Fingerprint: 96\%
& 
Face $+$ Voice $+$\newline
Fingerprint $=$\newline
98.5\%\\
\hline
Sanderson
(Sanderson and
Paliwal, 2002)
&
Face: PCA\newline
Voice: MFCC &
43 
& Time: 3 sessions (indoor)\newline
Noise addition to voice & 
Equal Error Rate\newline
Face: 10\%\newline
Voice: 12.41\%
&
Equal Error\newline
Rate 2.86\% \\
\hline
Proposed study & 
Face, voice:\newline
Commercial & 116 &
Location: Indoor and\newline
Outdoor (same day)\newline
Noise addition to eye\newline 
coordinates 
&
TARs at 1\% FAR\newline
Indoor-Outdoor\newline
Face: 80\%\newline
Voice: 67.5\%
&
TAR = 98\%\newline
at 1\% FAR\\
\hline
\multicolumn{6}{@{}l}{$^*$TAR--True Acceptance Rate\qquad 
$^{\#}$ FAR--False Acceptance Rate}
\end{tabular}
\end{table*}


\subsection{Tables, figures and schemes}
Graphics and tables may be positioned as they should appear in the
final manuscript. Figures, Schemes, and Tables should be numbered.
Structures in schemes should also be numbered consecutively, for ease
of discussion and reference in the text. \textcolor{newcolor}{\bf
Figures should be maximum half a page size.}

Depending on the
amount of detail, you can choose to display artwork in one column (20
pica wide) or across the page (42 pica wide). Scale your artwork in
your graphics program before incorporating it in your text. If the
artwork turns out to be too large or too small, resize it again in your
graphics program and re-import it. The text should not run along the
sides of any figure. This is an example for citation \citep{NewmanGirvan2004}.

\begin{figure}[!t]
\centering
\includegraphics[scale=.5]{cagfig01}
\caption{Studio setup for capturing face images indoor. Three light
sources L1, L2, L3 were used in conjunction with normal office lights.}
\end{figure}

You might find positioning your artwork within the text difficult
anyway. In that case you may choose to place all artwork at the end of
the text and insert a marker in the text at the desired place. In any
case, please keep in mind that the placement of artwork may vary
somewhat in relation to the page lay-out \citep{HullermeierRifqi2009}.

This can easily be achieved using \verb+endfloat.sty+ package. Please
refer the following documentation to use this package.
\makeatletter
\if@twocolumn
\begin{verbatim}
  http://mirrors.ctan.org/macros/latex/contrib/
  endfloat/endfloat.pdf
\end{verbatim}
\else
\begin{verbatim}
  http://mirrors.ctan.org/macros/latex/contrib/endfloat/endfloat.pdf
\end{verbatim}
\fi
\makeatother

\textcolor{newcolor}{\bf You should insert a caption for the figures
below the figures and for the tables the caption should be above the
tables.} 

Please remember that we will always also need highresolution versions
of your artwork for printing, submitted as separate files in standard
format (i.e. TIFF or EPS), not included in the text document. Before
preparing your artwork, please take a look at our Web page:
\url{http://www.elsevier.com/locate/authorartwork}.


\subsection{Lists}

For tabular summations that do not deserve to be presented as
a table, lists are often used. Lists may be either numbered or
bulleted. Below you see examples of both.
\begin{enumerate}
\item The first entry in this list
\item The second entry
\begin{enumerate}
\item A subentry
\end{enumerate}
\item The last entry
\end{enumerate}
\begin{itemize}
\item A bulleted list item
\item Another one
\end{itemize}

\subsection{Equations}
Conventionally, in mathematical equations, variables and
anything that represents a value appear in italics.
All equations should be numbered for easy referencing. The number
should appear at the right margin.
\begin{equation}
S_{\rm pg}'=\frac{S_{\rm pg}-\min(S_{\rm pG})}
 {\max(S_{\rm pG}-\min(S_{\rm pG})}
\end{equation}
In mathematical expressions in running text ``/'' should be used for
division (not a horizontal line). 

\section*{Acknowledgments}
Acknowledgments should be inserted at the end of the paper, before the
references, not as a footnote to the title. Use the unnumbered
Acknowledgements Head style for the Acknowledgments heading.

\section*{References}

Please ensure that every reference cited in the text is also present in
the reference list (and vice versa).

\section*{\itshape Reference style}

Text: All citations in the text should refer to:
\begin{enumerate}
\item Single author: the author's name (without initials, unless there
is ambiguity) and the year of publication;
\item Two authors: both authors' names and the year of publication;
\item Three or more authors: first author's name followed by `et al.'
and the year of publication.
\end{enumerate}
Citations may be made directly (or parenthetically). Groups of
references should be listed first alphabetically, then chronologically.

%%Vancouver style references.
\bibliographystyle{cag-num-names}
\bibliography{refs}

\section*{Supplementary Material}

Supplementary material that may be helpful in the review process should
be prepared and provided as a separate electronic file. That file can
then be transformed into PDF format and submitted along with the
manuscript and graphic files to the appropriate editorial office.

\end{document}

%%
